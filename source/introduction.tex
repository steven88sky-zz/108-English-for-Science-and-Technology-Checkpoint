\section{Introduction}
Hand gesture recognition has drawn increasingly interest
for numerous human-computer interface (HCI) applications
such as gaming control, aged or disabled care, sports fitness,
etc. Some early systems explored technologies such as com-
puter vision and capacitive sensor \cite{c1,c2,c3}, but their bulkiness
has limited their use to controlled environments \cite{c4} such
as research labs or hospitals. The emergence of low-power,
miniature hardware components with sensing, computation,
and communication capabilities has made possible a new
generation of a wireless devices such as a wrist watch \cite{c5},
wristband \cite{c6}, or smartphone, making ubiquitous computing a
reality.
\subsection{Sensing Technologies}
Most wearable HGRs to date perform inertial sensing (i.e.,
acceleration and rotation) as the primary modality for acquiring
gesture data. One reason is that accelerometers (ACC) and
gyroscopes are widely available in small sizes and in low
power. However, they are limited to detecting the trajectory
of the whole hand or forearm, such as drawing a circle or a
square, but fail to differentiate the different wrist activities.
Many commonly used hand gestures involve the movement of
the wrist. One example is that a hand-up gesture for ``stop``
generates nearly the same signal as a slight forearm shaking
but with smaller acceleration values in all three axes. Other
gestures may include a hand swing of directions, making a
fist, etc. These gestures are natural to users but can be difficult
to detect by an ACC.

\begin{figure}[t]
\centering
\includegraphics[width=8cm]{fig/fig1}
\caption{Proposed hardware system. Top left: optical sensor
board, top right: BLE and MPU-9250 board, bottom: battery.}
\end{figure}


Muscle tension can be sensed by surface electromyography
(sEMG) or optical muscle tension sensing (OMTS). The former
is most widely used but has not been amenable to wearable
HGRs. It requires a minimum of two electrodes, i.e., plus and
minus, to extract the differential signal generated by muscle
contraction. Many applications require three with an extra
ground electrode serving as a reference for the instrumentation
amplifier \cite{c7}. Commercial electrodes have to be wired and are
usually about 30 mm in diameter \cite{c8}. The ground electrode
is critical to the performance and has to be placed far away
from the target muscle to generate a stable reference. As an
alternative, we explore optical sensors similar to those used
in pulse oximeters for detecting oxygen concentration level
(SpO2) \cite{c9} and heart rate. It uses an LED to emit light into
human tissue and a PD to detect the reflected optical signal.
We validate the effectiveness of OMTS in terms of size, power
consumption, and wearing comfort.
\subsection{Proposed HANDIO System}
In this paper, we present HANDIO, for HAND gesture recognizer
based on Inertial and Optical muscle-tension sensing.
It consists of both the wearable hardware and the data-fusion
algorithm.

1) Hardware: The HANDIO hardware consists of an
OMTS, a triaxial accelerometer, and a Bluetooth Low Energy
(BLE) system-on-chip (SoC). The optical subsystem can be
made very small and lightweight by using an LED-PD combo
chip \cite{c10}. The overall dimension of the whole prototype system
is only 30 mm (L) $\times$ 15 mm (W) $\times$ 8 mm (H) with the largest
part being the 90 mAh lithium battery, as shown in Fig. 1, and
a production version can be made significantly more compact.
This novel miniaturized design enables the whole system to be
easily patched on the wrist area or integrated into a wristband
or watch without any extra sensor or wire.

To reduce the system power, the LED is controlled by fast switching pulse width modulation (PWM) signal. Our
experimental result also shows that the optical MTS is free
from baseline wondering and motion artifact. This feature
avoids the heavyweight filtering process and complicated pattern
recognition algorithms, thus further saving the system
power.

2) Fusion Algorithm: We also propose a fusion algorithm,
which includes rapid optical signal induced segmentation,
Dynamic Time Warp (DTW) \cite{c11}, and decision fusion. The
fusion of optical and acceleration data increases the recognizable
gestures. We tested a total of 8 wrist-involved hand
gestures with a detection rate over 93$\%$. The average power
consumption of the optical sensor is 258$\mu$W, which is only
0.4$\%$ of the overall system power.

The contribution of this work lies in the novel hardware
sensing platform design and the fusion algorithm. Even though
this work focuses only on the HGR, this system is also versatile
enough for detecting other joint or muscle activities such as
elbows and knees movement.
\subsection{Paper Outline}
This paper first provides a background on the OMTS sensor
and accelerometer-based hand gesture recognition systems. We
then describe the proposed sensing system and sensor-fusion
algorithm. Experimental results are analyzed with a discussion
of the implications. We also demonstrate the versatility of
the system by applying it to other joint movement detection.
Finally, we conclude with a summary and future works.