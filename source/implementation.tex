\section{IMPLEMENTATION}

\begin{figure}[t]
\centering
\subfigure[Cost matrix and warping window constraint]
{
  \includegraphics[width=4cm, scale=1]{fig/fig6a}
}
\subfigure[Slope constraint]
{
  \includegraphics[width=4cm, scale=1]{fig/fig6b}
}
\caption{Dynamic Time Warping with Constraints}
\end{figure}

\begin{figure}[t]
\centering
\includegraphics[width=8cm]{fig/fig7}
\caption{Signal matched with Dynamic Time Warping}
\end{figure}

\begin{figure}[b]
\centering
\includegraphics[width=8cm]{fig/fig8}
\caption{Proposed hybrid system}
\end{figure}

Fig. 8 shows the block diagram of the proposed HGR
system. It consists of a light emitter with a current source,
a photo diode (PD) with an OpAMP, an inertial sensor, and
a Bluetooth Low Energy (BLE) system-on-chip (SoC). To
quickly validate our design, we implement our sensor board
using a commercially available pulse-sensor design \cite{c21} but
customize the board to our design. For the accelerometer and
BLE board, we designed a prototype as shown in Fig. 1. Both
boards are connected using jumpers. The overall dimension of
the system including the battery is 30 mm (L) $\times$ 15 mm (W) $\times$
8 mm (H). The optical sensor board for prototyping purpose
may appear bulky but can be shrunk easily by fabricating a
flexible PCB, as the optical sensors themselves are small. Even
without optimization, the whole system is lightweight and can
be easily adhered onto the wrist area or further integrated into
a wearable sensing platform such as a wrist band or a watch.
The sensing location of the optical sensor is shown in Fig. 9.

\begin{figure}[t]
\centering
\includegraphics[width=6cm]{fig/fig9}
\caption{Sensing location on the wrist area.}
\end{figure}

\subsection{Optical Subsystem}
In the optical subsystem, the LED is driven by a PWMcontrolled
current source \cite{c22} for brightness control. The
LED current is limited to 500$\mu$A to prevent the on-surge
current during fast switching. The color of the LED is green
with a peak wavelength of 515 nm, which matches the peak
wavelength of the PD response curve. This way, the PD
response is maximized, thus further reducing the system power
consumption. The green LED color is also proved to be more
robust to motion artifact than other colors \cite{c23}. The LED
on-time is set to approximately 200$\mu$s in every 16 ms by
PWM control as shown in Fig. 11. In response to the LED,
the PD outputs a current that flows through the load resistor
to generates the voltage signal. The low-pass filter removes
noise and motion artifact in the voltage signal, and the OpAMP
amplifies it for sampling by the on-chip ADC on the CC2541.

\subsection{Accelerometer}
We use MPU-9250 \cite{c24} as the ACC controlled by the SPI
interface of CC2541. MPU-9250 is actually a 9 degree-off-reedom
(9-DoF) inertial sensor consisting of a triaxial ACC,
gyroscope, and compass. We use only the ACC for this test
while the gyro and compass are reserved for future use. The
typical operating current of the gyroscope is 3.2 mA and
only around 450$\mu$A for the ACC. The compass takes around
280$\mu$A and can be helpful to gesture recognition but it is
susceptible to the disturbance of other magnetic materials in
the vicinity \cite{c25}.


\subsection{Bluetooth Low Energy SoC}
CC2541 from TI is an integrated MCU and radio frequency
transceiver SoC that runs TI’s BLE protocol stack
\cite{c26} and the application code on a single chip. Industrystandard
profiles such as GATT (for attribute access) and GAP
(for connection and discovery) are already implemented. The
HGR system works as a slave server while the mobile device
(smartphone) works as a master client. The BLE stack runs
on a low-overhead, non-emptive, event-driven task executive
called OSAL, which implements tasks as call-back functions.

The CC2541 MCU performs MTS sampling, ACC reading,
and data transmission over BLE. The MCU remains in lowpower
mode normally. On each timer rollover, the MCU wakes
up, reads its ADC for the MTS signal, and reads the ACC data
from SPI. After all the data is acquired and stored into a local
buffer, the MCU queues a sending event in the scheduler and
go to low-power mode 3 (LPM3) again, from which the MCU
can be waken by interrupts only. The sending event will cause
the queued data to be sent the mobile client. This process
repeats at a rate of 62.5 Hz.