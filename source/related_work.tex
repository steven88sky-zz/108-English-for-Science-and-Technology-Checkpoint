\section{BACKGROUND AND RELATED WORKS}
Single ACC-based hand gesture recognition has been well
studied. One advantage of ACC is its availability on many
off-the-shelf portable devices. Examples include TI Chronos
Watch \cite{c5}, Nintendo Wii remote controller, and virtually all
smartphones. A digital ACC consumes less than 1 mW in
active mode. Acceleration data can be easily processed locally
or transferred through the wireless interface. On the software
side, simple filters such as low-pass filter (LPF) or median
filter \cite{c12} are usually used to smooth the signal before using
either feature-based or template-based detection algorithms to
detect a gesture. Feature-based methods use machine-learning
techniques such as Hidden Markov Model (HMM) or Support
Vector Machine (SVM) \cite{c3} to classify the signal according
to the extracted features. High accuracy is reported \cite{c13} but
a large training set is required to ensure high detection rate.
In template recognition, a widely used technique is dynamic
time warping (DTW). It can get started with only one template
for each gesture \cite{c14}. The signal is usually divided into fixedlength
segments, and dynamic programming is used to find

\tikzstyle{startstop} = [rectangle, rounded corners, minimum width=1cm, minimum height=1cm,text centered, draw=black, fill=red!30]

\tikzstyle{process} = [rectangle, minimum width=1cm, minimum height=1cm, text centered, text width=1cm, draw=black, fill=orange!30]

\tikzstyle{decision} = [diamond, minimum width=0.5cm, minimum height=0.5cm, text centered, text width=0.5cm, draw=black, fill=green!30]

\tikzstyle{arrow} = [thick,->,>=stealth]


\begin{tikzpicture}[node distance=1.8cm]
\node (start) [startstop] {\fontsize{8}{4}\selectfont Start};
\node (pro1) [process, right of=start] {\fontsize{5}{4}\selectfont Preprocessing \\ (ACC+MTS)};
\node (pro2) [process, right of=pro1] {\fontsize{5}{4}\selectfont Segmentation \\ (MTS Only)};
\node (dec1) [decision, right of=pro2] {\fontsize{3}{4}\selectfont Length \\ Checking};
\node (pro3) [process, below of=pro1] {\fontsize{5}{4}\selectfont Rejection};
\node (pro4) [process, below of=pro3] {\fontsize{5}{4}\selectfont Decision \\ Fusion};
\node (pro5) [process, right of=pro4] {\fontsize{5}{4}\selectfont 3D ACC DWT};
\node (dec2) [decision, right of=pro5] {\fontsize{3}{4}\selectfont Score$\tiny<$THD\_S};
\node (pro6) [process, right of=dec2] {\fontsize{5}{4}\selectfont 1D MTS DWT};
\node (stop) [startstop, left of=pro4] {\fontsize{8}{4}\selectfont End};
\draw [arrow] (start) -- (pro1);
\draw [arrow] (pro1) -- (pro2);
\draw [arrow] (pro2) -- (dec1);
\draw [arrow] (dec1) -| node[anchor=east] {YES} (pro6);
\draw [arrow] (dec1) |- node[anchor=south] {NO} (pro3);
\draw [arrow] (dec2) -- node {YES} (pro5);
\draw  (dec1) -- (dec2);
\draw [arrow] (pro6) -- (dec2);
\draw [arrow] (pro5) -- (pro4);
\draw [arrow] (pro3) -| (stop);
\draw [arrow] (pro4) -- (stop);
\end{tikzpicture}

\begin{figure}[h]
\centering
\caption{Detection Algorithm}
\end{figure}

the best matching subsequence with the preset templates. The
complexity of DTW is O(pST) with S being the segment size,
T the template size, and p the number of preset templates.
It can be easily implemented on a mobile platform such as a
smartphone and many modern microcontroller units (MCU).

To overcome the limitations of single ACC-based systems’
inability to detect wrist movements, multiple previous works
suggest to couple sEMG with ACC. In \cite{c15}, \cite{c16}, HMM, Kmeans
clustering, and decision fusion are used to fuse the
multi-channel sEMG signal with ACC. A set of more than
18 gestures can be detected with a detection rate of over
90$\%$. However, a multi-channel sEMG needs to be wired,
and electrodes are big in size and can be uncomfortable to
wear. Most applications require an extra troublesome step of
skin preparation \cite{c8}, which generally includes hair removal,
cleaning the skin, and applying conductive gel. The gel may
cause skin irritation and also introduces instability as the
gel dries over the time, thereby causing degradation of the
signal quality and baseline wondering. The closest one to our
system is \cite{c17}, which uses a single-channel sEMG with three
electrodes 30 mm in diameter. The muscle contraction signal
from sEMG is used as on-off control and only five gestures
can be detected. It is made wearable with a much smaller size
than the other two but still an order of magnitude larger than
our proposed system and cannot solve the electrode problem.
Therefore, ACC-sEMG-based systems are only suitable for a
controlled environment.

New sensing technologies, especially low-power wearable
sensors, provide potential solutions to this problem. Optical
sensor-based MTS is very promising in replacing sEMG in
this application due its size and wearing comfort. A wellknown
application of optical sensor is pulse oximetry, which
is acquired from an LED-PD pair. The LED emits light into
human tissue while the PD measures the transmittance or
reflectance of light. When used for muscle tension, only the
reflectance can be measured. It is a fairly complicated process
since the muscle movement has a triple effect: the muscle fiber
contraction changes both the light absorbance and the reflected
light path, while the blood volume in the muscle also affects
the reflected light. As a result, muscle movements manifest
as changes to the signal frequencies and amplitude in the PD
output. It is shown in \cite{c18} that optical sensors can differentiate
between isometric and isotonic contraction and in work \cite{c19},
\cite{c20}, similar optical sensors are used to monitor the upper
limb movement for controlling prosthetic limbs. In all these
works, the optical signal is only used for on-off control. Our
study shows that far more details are provided even in a singlechannel
MTS. We will show how to utilize the MTS signal in
template matching.